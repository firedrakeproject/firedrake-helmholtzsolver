\documentclass[12pt]{article}
\usepackage{amsmath,amssymb}
\usepackage[cm]{fullpage}
\usepackage{bbm}
\renewcommand{\vec}[1]{\boldsymbol{#1}}
\newcommand{\grad}{\operatorname{grad}}
\newcommand{\divergence}{\operatorname{div}}
\newcommand{\intOmega}[1]{\langle#1\rangle_{\Omega}}
\newcommand{\Vpressure}{\mathbb{V}_2}
\newcommand{\Vvelocity}{\mathbb{V}_1}
\newcommand{\VvelocityTr}{\mathbb{V}^*_1}
\newcommand{\indexSet}{\mathcal{I}}
\newcommand{\grid}{\mathcal{T}}
\usepackage{algorithm,algorithmic}
\title{Ideas for a matrix free mixed-finite element multigrid solver}
\author{Eike M\"{u}ller, University of Bath}
\begin{document}
\maketitle
%%%%%%%%%%%%%%%%%%%%%%%%%%%%%%%%%%%%%%%%%%%%%%%%%%%%
\section{Elliptic operator}
%%%%%%%%%%%%%%%%%%%%%%%%%%%%%%%%%%%%%%%%%%%%%%%%%%%%
Pressure / velocity system arising from implicit time stepping
\begin{equation}
 \begin{aligned}
  \phi + \omega \divergence\left(\phi^* \vec{u}\right) &= r_{\phi} \\
  \vec{u} + \omega \grad\phi &= \vec{r}_{\vec{u}}
 \end{aligned}
  \label{eqn:MixedSystem}
\qquad 
\phi,r_\phi \in \Vpressure,\;\;
\vec{u},\vec{r}_{\vec{u}}\in\Vvelocity,\;\;
\phi^* \in\VvelocityTr
\end{equation}
with $\omega = \kappa \Delta t$. The system is solved on the surface of a sphere $\Omega$, so there are no boundary conditions. Choose $P_0$ (or $Q_0$ on quads) elements for the discontinuous pressure space $\Vpressure$ and $RT_0$ elements for the velocity space $\Vvelocity$. $\phi^*$ is a mapping $\Vvelocity\rightarrow\Vvelocity$ and lives in the space $\VvelocityTr$ (that's probably not the correct mathematical notation. Is this (equivalent to) the trace space?). Explicitly
\begin{xalignat}{3}
  \phi &= \sum_{\operatorname{cells}\;i} \Phi_i \beta_i(x) \in P_0\;\text{or}\;Q_0\equiv \Vpressure, &
  \vec{u} &= \sum_{\operatorname{edges}\;e} U_e \vec{v}_e(x) \in RT_0 \equiv\Vvelocity, &
  \phi^* &= \sum_{\operatorname{edges}\;e} \Phi^*_{ee}\gamma_e(x) \in \VvelocityTr.
\end{xalignat}
\paragraph{Basis functions.}
\begin{itemize}
  \item $\beta_i(x)$ is 1 in the cell $i$ and zero everywhere else
	\item $\gamma_e(x)$ is 1 on the edge $e$ and 0 everywhere else
  \item $\vec{v}_e(x)$ are the $RT_0$ basis functions with unit flux through edge $e$.
\end{itemize}
$\Phi^*$ is a diagonal matrix. In the following I always use indices $e,e'$ for edges and $i,j,k$ for cells.
Multiply (\ref{eqn:MixedSystem}) by test functions $\psi\in \Vpressure$, $\vec{w}\in\Vvelocity$ and integrate over the domain $\Omega$ to obtain 
\begin{equation}
 \begin{aligned}
  \intOmega{\psi \phi} + \omega \intOmega{\psi \divergence(\phi^* \vec{u})} &= 
  \intOmega{\psi r_{\phi}} \\
  \intOmega{\vec{w}\cdot\vec{u}} - \omega \intOmega{\divergence \vec{w} \phi} &= \intOmega{\vec{w}\cdot \vec{r}_{\vec{u}}}
 \end{aligned}
 \qquad\forall{\psi\in\Vpressure, \vec{w}\in\Vvelocity}
\end{equation}
where $\intOmega{\cdot}$ denotes integration over the entire domain $\Omega$.
This leads to the matrix system
\begin{equation}
 \begin{aligned}
  M_{\phi} \vec{\Phi} + \omega B (\Phi^* \vec{U}) &= M_{\phi}\vec{R}_\phi \\
  M_{u} \vec{U} - \omega B^T \vec{\Phi} &= M_u \vec{R}_{\vec{u}}
 \end{aligned}
\label{eqn:MatrixSystem}
\end{equation}
with the mass matrices
\begin{equation}
 \begin{aligned}
  \left(M_{\phi}\right)_{ij} &\equiv \int_{\Omega} \beta_{i}(x)\beta_j(x)\;dx = \delta_{ij} \int_{E_i} dx = \delta_{ij} |E_i| \\
  \left(M_{u}\right)_{ee'} &\equiv \int_{\Omega} \vec{v}_e(x)\cdot \vec{v}_{e'}(x)\; dx
 \end{aligned}
\end{equation}
We write $|E_i|$ for the volume of cell $i$. The discrete derivative matrix is given by
\begin{equation}
  B_{ie} \equiv \int_{\Omega} \divergence \vec{v}_e(x) \beta_i(x)\; dx = \int_{E_i} \divergence \vec{v}_e(x) \;dx = \int_{\partial E_i} \vec{v}_e(x)\cdot \hat{\vec{n}} \; dx = \pm 1,
\end{equation}
the sign depends on whether the the flux basis function $\vec{v}_e(x)$ on edge $e$ points into cell $i$ ($B_{ie} = -1$) or out of it ($B_{ie}=+1$).
%%%%%%%%%%%%%%%%%%%%%%%%%%%%%%%%%%%%%%%%%%%%%%%%%%%%
\section{Incremental iterative solver}
%%%%%%%%%%%%%%%%%%%%%%%%%%%%%%%%%%%%%%%%%%%%%%%%%%%%
Solve the pressure/velocity system in (\ref{eqn:MatrixSystem}) iteratively as follows (the scheme is the same as the nested iteration in \cite{Wilson2010}):
The solution at iteration $k$ is given by $\vec{\Phi}^{(k)}$, $\vec{U}^{(k)}$.
 Use this pressure and velocity to calculate residuals
\begin{equation}
 \begin{aligned}
  M_{\phi} \vec{\Phi}^{(k)} + \omega B (\Phi^* \vec{U}^{(k)}) &= M_{\phi}\vec{R}^{(k)}_\phi \\
  M_{u} \vec{U}^{(k)} - \omega B^T \vec{\Phi}^{(k)} &= M_u \vec{R}^{(k)}_{\vec{u}}
 \end{aligned}
\end{equation}
To obtain the next iterate $\vec{\Phi}^{(k+1)} = \vec{\Phi}^{(k)}+\vec{\Phi}'$, $\vec{U}^{(k+1)} = \vec{U}^{(k)}+\vec{U}'$, find the increments $\vec{\Phi}'$, $\vec{U}'$ which solve the residual correction equation
\begin{equation}
 \begin{aligned}
  M_{\phi} \vec{\Phi}' + \omega B (\Phi^* \vec{U}') &= M_{\phi}\vec{R}'_\phi
= M_\phi\left(\vec{R}_\phi-\vec{R}^{(k)}_\phi\right)\\
  M^*_{u} \vec{U}' - \omega B^T \vec{\Phi}' &= M^*_u \vec{R}'_{\vec{u}}
= {M_u} \left(\vec{R}_{\vec{u}}-\vec{R}^{(k)}_{\vec{u}}\right)
 \end{aligned}
\label{eqn:MatrixSystemCorrection}
\end{equation}
where we replaced $M_u$ by the lumped mass-matrix $M_u^*$ (see appendix \ref{sec:MassLumping}). By solving the second equation in (\ref{eqn:MatrixSystemCorrection}) for $\vec{U}'$ and inserting it into the first, one obtains the following elliptic equation for the pressure correction $\vec{\Phi}'$:
\begin{equation}
  M_{\phi}\vec{\Phi}' + \omega^2 B \alpha^{(u)} B^T \vec{\Phi}'
  \equiv H\vec{\Phi}' = M_{\phi}\vec{F} \equiv M_{\phi}\vec{R}'_{\phi}-\omega B \Phi^* \left(M_u^*\right)^{-1}M_u\vec{R}'_{\vec{u}}
\label{eqn:EllipticEquation}
\end{equation}
where we defined the diagonal matrix
\begin{equation}
  \alpha^{(u)} \equiv \Phi^* \left(M_u^*\right)^{-1}.
\end{equation}
which we assume to be positive definite. We use a multigrid algorithm to solve 
(\ref{eqn:EllipticEquation}) approximately for $\vec{\Phi}'$ and then calculate the velocity correction as 
\begin{equation}
  \vec{U}' = \left(M_u^*\right)^{-1}\left(M_u\vec{R}'_{\vec{u}}+ \omega B^T \vec{\Phi}'\right).
\end{equation}
%%%%%%%%%%%%%%%%%%%%%%%%%%%%%%%%%%%%%%%%%%%%%%%%%%%%
\subsection{Schur complement preconditioner}
%%%%%%%%%%%%%%%%%%%%%%%%%%%%%%%%%%%%%%%%%%%%%%%%%%%%
More abstractly, we can write the original mixed system in (\ref{eqn:MatrixSystem}) as
\begin{equation}
  A\vec{X} = \vec{R}
\end{equation}
where $\vec{X}=(\vec{\Phi},\vec{U})^T$ is the state vector and 
$\vec{R}=(M_{\phi}\vec{R}_{\phi},M_u\vec{R}_{\vec{u}})^T$ is the right hand side; the matrix $A$ is given by
\begin{equation}
  A \equiv
\begin{pmatrix}
  M_\phi & \omega B\Phi^* \\
  -\omega B^T & M_u
\end{pmatrix}
\label{eqn:MixedMatrixA}
\end{equation}
The incremental solver is then a preconditioned Richardson iteration
\begin{equation}
  \vec{X}^{(k+1)} = \vec{X}^{(k)} + P^{-1}\left(\vec{R}-A\vec{X}^{(k)}\right)
\end{equation}
and the preconditioner is the original matrix in (\ref{eqn:MixedMatrixA}) with the velocity mass matrix $M_u$ replaced by its lumped version $M_u^*$:
\begin{equation}
    P \equiv
\begin{pmatrix}
  M_\phi & \omega B\Phi^* \\
  -\omega B^T & M^*_u
\end{pmatrix}
\end{equation}
The inverse of $P$ can be written as
\begin{equation}
    P^{-1} =
\begin{pmatrix}
  \mathbbm{1} & 0 \\
 (M_u^*)^{-1}\omega B^T  & \mathbbm{1}
\end{pmatrix}
\begin{pmatrix}
  H^{-1} & 0 \\
  0 & (M_u^*)^{-1}
\end{pmatrix}
\begin{pmatrix}
  \mathbbm{1} & -\omega B\Phi^* (M_u^*)^{-1} \\
  0 & \mathbbm{1}
\end{pmatrix}
\end{equation}
with the Schur complement pressure operator $H$ as defined in (\ref{eqn:EllipticEquation}).
%%%%%%%%%%%%%%%%%%%%%%%%%%%%%%%%%%%%%%%%%%%%%%%%%%%%
\section{Multigrid solver}
%%%%%%%%%%%%%%%%%%%%%%%%%%%%%%%%%%%%%%%%%%%%%%%%%%%%
Solve the elliptic equation in (\ref{eqn:EllipticEquation}) with a multigrid iteration, either as a standalone solver or as a preconditioner for CG.
%%%%%%%%%%%%%%%%%%%%%%%%%%%%%%%%%%%%%%%%%%%%%%%%%%%%
\subsection{Smoother}
%%%%%%%%%%%%%%%%%%%%%%%%%%%%%%%%%%%%%%%%%%%%%%%%%%%%
For the smoother we use the following Richardson iteration
\begin{equation}
  \vec{\Phi} \mapsto \vec{\Phi} + 2\mu D^{-1} \left(\vec{F}-H\vec{\Phi}\right)
  \label{eqn:Richardson}
\end{equation}
where $D$ is a diagonal matrix which fulfils the following conditions:
\begin{enumerate}
  \item $D$ is positive definite 
  \item $\mu H < D$ (i.e. $\mu \vec{\Psi}^T H \vec{\Psi} < \vec{\Psi}^T D \vec{\Psi}$ for all $\vec{\Psi}$)
  \item The iteration in (\ref{eqn:Richardson}) should be efficient at reducing high-frequency error components.
\end{enumerate}
The first two conditions guarantee that the spectrum of the iteration matrix 
\begin{equation}
  S \equiv \mathbb{I} - 2 \mu D^{-1} H,
\end{equation}
which describes the reduction of the error $\vec{\Psi}-\vec{\Psi}_{\operatorname{exact}}$, is contained in $[-1,1]$ and the iteration does not diverge.
%%%%%%%%%%%%%%%%%%%%%%%%%%%%%%%%%%%%%%%%%%%%%%%%%%%%
\subsubsection{Lowest order}
%%%%%%%%%%%%%%%%%%%%%%%%%%%%%%%%%%%%%%%%%%%%%%%%%%%%
We first treat the lowest order ($DG_0$+$RT_0$) case.
To bound $H$, consider the term $B\alpha^{(u)} B^T$. We can write
\begin{equation}
 \begin{aligned}
  \vec{\Psi} B \alpha^{(u)} B^T \vec{\Psi} &= \sum_{\operatorname{cells}\;i,j}\;\;\sum_{\operatorname{edges}\;e} \Psi_i B_{ie} \alpha^{(u)}_{ee} B_{je} \Psi_j 
\\
  &= \sum_{e} \alpha_e^{(u)} \left(\Psi_{i_+(e)}-\Psi_{i_-(e)}\right)^2
  & \le 2\sum_e\left(\alpha_e^{(u)} \Psi_{i_+(e)}^2+\alpha_e^{(u)} \Psi_{i_-(e)}^2\right)\\
  &= 2 \sigma \sum_i \alpha_{ii}^{(\phi)} \Psi_i ^2 = 2\sigma \vec{\Psi}^T \alpha^{(\phi)}\vec{\Psi}
 \end{aligned}\label{eqn:MatrixBound}
\end{equation}
In this expression $i_{\pm}(e)$ denotes the two cells adjacent to the edge $e$. The inequality follows from $(a-b)^2 \le 2(a^2+b^2)$. The integer $\sigma$ denotes the number of edges adjacent to a cell, i.e. $\sigma=3$ for triangles and $\sigma=4$ for quadrilaterals. We define the projection of $\alpha^{(u)}$ onto the  cells as 
\begin{equation}
  \alpha^{(\phi)}_{ii} \equiv \frac{1}{\sigma} \sum_{e'\in e(i)} \alpha_{e'e'}^{(u)}
\end{equation}
where $e(i)$ are all edges touching cell $i$ and $\alpha^{(\phi)}$ is a diagonal matrix.

The bound in (\ref{eqn:MatrixBound}) is sharp for a regular triangulation/tiling of a flat domain. To see this, consider the high-frequency field which is given by $\Psi_i = \pm 1$ and changes sign whenever an edge is crossed. On a regular flat grid this field exists and $(\Psi_{i_+(e)-\Psi_{i_-(e)}})^2 = 2 (\Psi_{i_+(e)}^2+\Psi_{i_-(e)}^2) = 4$ for all edges $e$. For the semi-structured grids on the sphere the bound will still be reasonably sharp, at least for the fine grids where we can choose such an oscillating field on each of the panels - it will only be violated where the panels meet, i.e. on a 1d subdomain.

We therefore use the following diagonal matrix $D$ in the smoother:
\begin{equation}
  D = M_{\phi} + 2\omega^2 \sigma \alpha^{(\phi)}
\label{eqn:blockSmootherLO}
\end{equation}
The spectrum of the error reduction matrix $S$ is contained in
\begin{equation}
  \lambda \in [1-2\mu,1]
\end{equation}
where the high-frequency modes (for which $H\Psi^{(hf)}\approx D\Psi^{(hf)}$) have eigenvalues of $\lambda \approx 1-2\mu$. A possible choice would be $\mu=2/3$.
%%%%%%%%%%%%%%%%%%%%%%%%%%%%%%%%%%%%%%%%%%%%%%%%%%% 
\paragraph{Implementation.}
%%%%%%%%%%%%%%%%%%%%%%%%%%%%%%%%%%%%%%%%%%%%%%%%%%% 
To apply the matrix $D^{-1}$ to a field $\vec{\Phi}$, i.e. $\vec{\Psi}=D^{-1}\vec{\Phi}$, loop over all cells and in each cell
\begin{enumerate}
  \item calculate
\begin{equation}
  D_{ii} = |E_i| + 2\omega^2 \sum_{e'\in e(i)} \alpha^{(u)}_{e'e'}
\end{equation}
  \item calculate $\Psi_i = \Phi_i/D_{ii}$.
\end{enumerate}
%%%%%%%%%%%%%%%%%%%%%%%%%%%%%%%%%%%%%%%%%%%%%%%%%%%%
\subsubsection{Higher order}
%%%%%%%%%%%%%%%%%%%%%%%%%%%%%%%%%%%%%%%%%%%%%%%%%%%%
The higher order case $DG_1$ + $BDFM_1$ can be treated similarly. Her we only consider the case of constant reference profiles, i.e. $\alpha^{(u)}=\left(M_u^*\right)^{-1}$.
To bound the term $\vec{\Phi}^T B\alpha^{(u)}$ we write as in Eqn. (\ref{eqn:MatrixBound})
\begin{equation}
  \vec{\Psi} B \alpha^{(u)} B^T \vec{\Psi} = \sum_{\operatorname{cells}\;i,j}\;\;\sum_{\operatorname{edges}\;e} \vec{\Psi}_i B_{ie} \alpha^{(u)}_{ee} B^T_{ej} \vec{\Psi}_j 
\label{eqn:MatrixBoundBDFM1}
\end{equation}
with the only difference being that now $\vec{\Psi}_i$ is the vector of the three degrees of freedom on cell $i$ and $\alpha^{(u)}_{ee}$ is a $4\times 4$ matrix. The $3\times 4$ matrices $B_{ie}$ arise as follows: consider the finite element discretisation of the divergence operator:
\begin{equation}
  \int_{\Omega} \psi^{(j)}\operatorname{div}\vec{w}^{(k)}\; dx
\end{equation}
where $\vec{w}^{(k)}$ and $\psi^{(j)}$ are basis function of the $BDFM_1$ and $DG_1$ spaces respectively. The integral only vanishes if the $BDFM_1$ dofs with indices $j$ are associated with cell the $E_i$ on which $\psi^{(j)}$ is nonzero. Hence define the $3\times 9$ matrix $\tilde{B}_i$ on each cell $i$ as
\begin{equation}
  \left(\tilde{B}_i\right)_{jk} \equiv \int_{E_i} \psi^{(g_i(j))}\operatorname{div}\vec{w}^{(g_i(k))}\; dx
\end{equation}
where for each cell the function $g_i$ maps from the local dofs $j\in1,\dots,4$ and $k\in1,\dots,9$ to the global dofs in the $DG_1$ and $BDFM_1$ spaces. This matrix can be set up in firedrake.
In total for each cell there are 9 indices $k$ for which this is the case, three associated with each facet $e$ of the cell. For each cell $i$ and facet $e$ we hence define the elements of the $3\times 4$ matrix $B_{ie}$ as
\begin{equation}
  (B_{ie})_{jk} \equiv \int_{E_i} \psi^{(g_i(j))} \operatorname{div}\vec{w}^{(h_e(k))}\; dx.
\end{equation}
Here $j\in1,2,3$ are the local indices of the $DG_1$ dofs in cell $i$ and $k\in1,2,3,4$ the local indices of the $BDFM_1$ dofs associated with edge $e$; the functions $g_i$ and $h_e$ map these to the corresponding global indices of the dofs in the $DG_1$ and $BDFM_1$ spaces. Note that for each cell one column of the matrix $B_{ie}$ is zero since the integral vanishes for the tangential dofs which is not associated with  this cell.

For a given edge $e$ and cell $i$ the matrices $B_{ie}$ and $\tilde{B}_i$ are related as follows:
\begin{equation}
  \left(B_{ie}\right)_{jk} = \begin{cases}
  0 & \text{if dof $h_e(k)$ is not associated with cell $i$} \\
  \left(\tilde{B}_i\right)_{j\omega_{ie}(k)} & \text{otherwise}
\end{cases}
\end{equation}
The function $\omega_{ie}$ maps the local index on a facet $e$ (in range $1,\dots,4$) to the corresponding local index (in range $1,\dots,9$) on cell $i$.

We can now use the identity
\begin{equation}
\left(\vec{x}+\vec{y}\right)^T A \left(\vec{x}+\vec{y}\right)
\le 2\left(\vec{x}^T A\vec{x}+\vec{y}^T A\vec{y}\right)
\end{equation}
which holds for any positive definite symmetric matrix $A$ to bound the expression in Eqn. (\ref{eqn:MatrixBoundBDFM1}) as follows:
\begin{equation}
 \begin{aligned}
  \vec{\Psi} B \alpha^{(u)} B^T \vec{\Psi}
  &= \sum_{e} \left(\vec{\Psi}^T_{i_+(e)}B_{i+(e),e}+\vec{\Psi}^T_{i_-(e)}B_{i-(e),e}\right)\alpha^{(u)}_{ee}
\left(B^T_{e,i+(e)}\vec{\Psi}_{i_+(e)}+B^T_{e,i-(e)}\vec{\Psi}_{i_-(e)}\right)\\
  & \le 2\sum_e
\left(\vec{\Psi}^T_{i_+(e)}B_{i+(e),e}\alpha^{(u)}_{ee}B^T_{e,i+}\vec{\Psi}_{i_+(e)}
+\vec{\Psi}^T_{i_-(e)}B_{i-(e),e}\alpha^{(u)}_{ee}B^T_{e,i-}\vec{\Psi}_{i_-(e)}
\right)
\\
  &= 2 \sigma \sum_i \vec{\Psi}^T_i\alpha^{(\phi)}_{ii}\vec{\Psi}_i = 2\sigma \vec{\Psi}^T \alpha^{(\phi)}\vec{\Psi}
 \end{aligned}
\end{equation}
In this equation the local $3\times 3$ blocks of the block-diagonal matrix $\alpha^{(\phi)}$ is given as
\begin{equation}
  \alpha_{ii}^{(\phi)} \equiv \frac{1}{\sigma}\sum_{e'\in e(i)} B_{ie'}\alpha^{(u)}_{e'e'}B^T_{e'i}
\end{equation}
with $\sigma$ as defined above. 
Together with the local $DG_1$ mass matrix we can use the following block-diagonal matrix in the smoother:
\begin{equation}
  D = M_{\phi} + 2\omega^2 \sigma \alpha^{(\phi)}.
\end{equation}
%%%%%%%%%%%%%%%%%%%%%%%%%%%%%%%%%%%%%%%%%%%%%%%%%%% 
\subsection{Intergrid operators}
%%%%%%%%%%%%%%%%%%%%%%%%%%%%%%%%%%%%%%%%%%%%%%%%%%%%
%%%%%%%%%%%%%%%%%%%%%%%%%%%%%%%%%%%%%%%%%%%%%%%%%%%%
\subsubsection{Residual and coarse grid correction}
%%%%%%%%%%%%%%%%%%%%%%%%%%%%%%%%%%%%%%%%%%%%%%%%%%%%
In the multigrid algorithm, the residual needs to be restricted to the coarse grid (and becomes the RHS there). The coarse grid correction has to be prolongated to the fine grid and added to the fine grid solution. We need the following operations:
\begin{xalignat}{2}
  \phi^{(H)} &= I_{h}^H \phi^{(h)} &
  \phi^{(h)} &= I_{H}^h \phi^{(H)} \qquad \text{with}\;\;
  \phi^{(h)} \in \Vpressure^{(h)}, \phi^{(H)} \in \Vpressure^{(H)}\subset\Vpressure^{(h)}
\end{xalignat}
For the restriction, average the values over all four children cells to obtain the value in the father cell:
\begin{eqnarray}
  \Phi_i^{(H)} = \frac{1}{4} \sum_{i'\in c(i)} \Phi_{i'}^{(h)}\qquad \text{with}\;\;c(i) \equiv \{i'\in \indexSet_E^{(h)} :E_{i'}\subset E_{i}\}
\label{eqn:restriction}
\end{eqnarray}
Denote by $\indexSet_E^{(h)}$ the set of cell indices on the grid $\grid^{(h)}$. For each fine grid cell $i$ we can also define a map to the father cell $f(i) \equiv i'\in \indexSet_E^{(H)} : E_i \subset E_{i'}$.

The simplest choice for the prolongation is injection, i.e. $\phi^{(h)} = I_H^h \phi^{(H)} = \phi^{(H)}$. Explicitly this choice corresponds to
\begin{equation}
  \Phi^{(h)}_i = \Phi^{(H)}_{f(i)}
\end{equation}

Then prolongation and restriction are transpose of each other
\begin{equation}
  \intOmega{\psi^{(h)}I_H^h \phi^{(H)}} = \intOmega{\psi^{(h)}\phi^{(H)}} = \intOmega{I_h^H \psi^{(h)}\phi^{(H)}} \qquad\text{for all}\;\; \psi^{(h)}\in \Vpressure^{(h)}, \phi^{(H)} \in\Vpressure^{(H)}.
\end{equation}
It might be necessary to use a higher order prolongation, for example linear interpolation, see the discussion in \cite{Hemker1990}.

To implement this prolongation and restriction we need to store:
\begin{itemize}
  \item For each fine grid cell $i$ the map $f(i)$ to its father cell
  \item For each coarse grid cell the set $c(i)$ of its children
\end{itemize}
%%%%%%%%%%%%%%%%%%%%%%%%%%%%%%%%%%%%%%%%%%%%%%%%%%%%
\subsubsection{Coarse grid matrix}
%%%%%%%%%%%%%%%%%%%%%%%%%%%%%%%%%%%%%%%%%%%%%%%%%%%%
We also need to construct the elliptic operator (\ref{eqn:EllipticEquation}) on the coarse grid. For this we use the normal $\Vpressure$ mass matrix and finite difference operator $B$. We also need $\alpha^{(u)}$ on the coarse grid. For this we could just average the field over the edge, i.e. set
\begin{equation}
  \left(\alpha^{(u)}\right)_{ee} = \frac{1}{2} \sum_{e'\in c(e)}\alpha^{(u)}_{e'e'} \qquad\text{with}\;\;c(e) \equiv \{e'\in \indexSet_S^{(h)}:S_{e'}\subset S_{e}\}
\end{equation}
Here $\indexSet_S^{(h)}$ is the set of edge indices and $c(e)$ are the children of edge $e$.
\appendix
%%%%%%%%%%%%%%%%%%%%%%%%%%%%%%%%%%%%%%%%%%%%%%%%%%%%
\section{Mass lumping}\label{sec:MassLumping}
%%%%%%%%%%%%%%%%%%%%%%%%%%%%%%%%%%%%%%%%%%%%%%%%%%%%

%%%%%%%%%%%%%%%%%%%%%%%%%%%%%%%%%%%%%%%%%%%%%%%%%%%%
\subsection{Lowest order ($RT0$)}
%%%%%%%%%%%%%%%%%%%%%%%%%%%%%%%%%%%%%%%%%%%%%%%%%%%%
Use John's mass lumping \cite{Thuburn2013}, i.e for a particular edge $e$ require that the diagonal mass matrix gives the same result as the full mass matrix when acting on a solid body rotation flow field passing through this edge. Let $\vec{u}_{(1)}$, $\vec{u}_{(2)}$ and $\vec{u}_{(3)}$ be the solid body rotation flow fields for rotations around the $x-$, $y-$ and $z-$ axis\footnote{The corresponding stream functions are $\psi_{(1)} = x$, $\psi_{(2)}= y$ and $\psi_{(3)}=z$.} and project $\vec{u}_{(k)}$ onto $\Vvelocity$, i.e. $\vec{u}_{(k)} = \sum_e \left(U_{(k)}\right)_e \vec{u}_e(x)$. 
Then the entries of the lumped mass matrix are given by
\begin{equation}
  \left(M^*_{u}\right)_{ee} = \frac{\sum_{k=1}^3 \left(U_{(k)}\right)_e\left(V_{(k)}\right)_e}{\sum_{k=1}^3 \left(U_{(k)}\right)^2_e} \qquad
  \text{where}\quad \vec{V}_{(k)} \equiv M_{u} \vec{U}_{(k)}.
\end{equation}
This needs to be calculated once and a scalar stored for each edge.

%%%%%%%%%%%%%%%%%%%%%%%%%%%%%%%%%%%%%%%%%%%%%%%%%%%%
\subsection{Higher order ($BDFM1$)}
%%%%%%%%%%%%%%%%%%%%%%%%%%%%%%%%%%%%%%%%%%%%%%%%%%%%
For higher order finite element velocity spaces we proceed similarly. For the $BDFM1$ space four degrees of freedom are associated with each triangle facet. Two of these correspond to continuous normal fluxes which are shared between neighbouring cells whereas the other two are the discontinuous tangential fluxes which are associated to each cell.
The lumped mass matrix is block-diagonal, with each $4\times4$ block associated to a particular facet. We construct the block matrix on a facet $e$ by demanding that when it is applied to certain solid body rotation fields it gives similar results as the full mass matrix. Note that since we also require the lumped mass matrix to be symmetric and positiv definite, it is not possible to achieve exact agreement on a particular edge.

For the following construction we use the following solid body rotation fields in Tab. \ref{tab:SBR}, which represent rotations around the different coordinate axes. For the fields $\vec{u}_{(i)}$ both hemispheres rotate in the same directions, whereas they rotation in opposite directions for $\tilde{\vec{u}}_{(i)}$.
\begin{table}
 \begin{center}
  \begin{tabular}{|l|lll|}
    \hline
    Type A &
    $\vec{u}_{(1)}=(0,-z,y)$ & 
	 $\vec{u}_{(2)}=(z,0,-x)$ & 
	 $\vec{u}_{(3)}=(-y,x,0)$ \\
    \hline
	 Type B &
    $\tilde{\vec{u}}_{(1)}=(0,-zx,yx)$ & 
	 $\tilde{\vec{u}}_{(2)}=(zy,0,-xy)$ & 
	 $\tilde{\vec{u}}_{(3)}=(-yz,xz,0)$ \\
	 \hline
  \end{tabular}
  \caption{Solid body rotation fields used for the $BDFM1$ mass lumping construction}
  \label{tab:SBR}
 \end{center}
\end{table}
Further let $\vec{n}^{(e)}$ and $\vec{t}^{(e)}$ denote the normalised normal and tangent vectors on edge $e$. We then use the fields in Tab. \ref{tab:SBR} to construct the following three velocity fields:
\begin{enumerate}
  \item Solid body rotation of type A with maximal flow through the facet $e$: $\vec{u}^{(1)} = \sum_{i=1}^3 t^{(e)}_i \vec{u}_{(i)}$
  \item Solid body rotation of type A with maximal flow parallel to the facet $e$: $\vec{u}^{(2)} = \sum_{i=1}^3 n^{(e)}_i \vec{u}_{(i)}$
  \item Solid body rotation of type B with maximal flow through the facet $e$: $\vec{u}^{(3)} = \sum_{i=1}^3 t^{(e)}_i \tilde{\vec{u}}_{(i)}$
\end{enumerate}
It might also be possible to use further solid body rotation fields to constrain the lumped mass matrix, and for generality we denote the number of solid body fields used in this construction by $\nu$, i.e. $\nu=3$ in our case.
Denote with $\vec{U}^{(k)}$ the dof vectors of these fields, and $\vec{V}^{(k)}=M_u\vec{U}^{(k)}$.
Then we demand that the following matrix function is minimised by $(M_u^*)_{ee}$ the lumped mass matrix on facet $e$:
\begin{equation}
  f(A) = \sum_{k=1,\nu}||\left(A\vec{U}^{(k)}-\vec{V}^{(k)}\right)_e||_2^2
\label{eqn:minimiserFunction}
\end{equation}
where $A$ is symmetric positive definite. We might make additional assumptions on $A$, more restrictive choices could for example be:
\begin{itemize}
  \item $A$ is proportional to the $4\times 4$ identity matrix
  \item $A$ is diagonal
\end{itemize}
To minimise the function in (\ref{eqn:minimiserFunction}) under these constraints, write
\begin{equation}
  A \equiv \sum_\mu a_\mu E_\mu
\end{equation}
where $E_\mu$ are $4\times 4$ basis matrices which fulfill the constraints. For example, if $A$ is assumed to be diagonal, a possible choice would be the matrices with $(E_\mu)_{ij} = \delta_{\mu i}\delta{ij}$, i.e. the matrices which have a 1 on the $\mu$-th diagonal entry. For a general symmetric matrix there are 10 basis matrices, for a diagonal one there are 4 and if we require the lumped mass matrix to be proportional to the identity matrix there is only 1 (the $4\times 4$ identity matrix itself).

In the most general case the coefficients $a_\mu$, and hence the lumped mass matrix, can be found by solving the system
\begin{equation}
  b_{\rho\sigma} a_{\sigma} = r_{\rho}
  \label{eqn:massMatrixSystem}
\end{equation}
where
\begin{xalignat}{2}
  b_{\rho\sigma} &\equiv \sum_{k=1}^\nu \left(\vec{U}^{(k)}\right)^T
  E_{\rho}E_{\sigma}\vec{U}^{(k)} &
  r_{\rho} &\equiv \left(\vec{U}^{(k)}\right)^T
  E_{\rho}\vec{V}^{(k)}
\end{xalignat}
For special cases this can obviously be simplified significantly since most of the entries of the matrix $b$ can be zero.

In firedrake this is implemented as follows:
\begin{itemize}
  \item Project the fields $\vec{u}_{(i)}$ and $\tilde{\vec{u}}_{(i)}$ onto the $BDFM1$ space
  \item Calculate the fields $\vec{V}_{(i)} = M_u \vec{U}_{(i)}$ and $\tilde{\vec{V}}_{(i)} = M_u \tilde{\vec{U}}_{(i)}$ by multiplying $\vec{u}_{(i)}$ and $\tilde{\vec{u}}_{(i)}$ with a test function and integrating over space.
  \item Loop over the facets, and using the normal and tangential vectors there, calculate the local fields $\vec{U}^{(k)}$, $\vec{V}^{(k)}$ as linear combinations of these fields.
  \item On each facet solve the system in Eqn. (\ref{eqn:massMatrixSystem}) to calculate the local lumped mass matrix block.
\end{itemize}
\nocite{Verfurth1984}
\nocite{Ewing1990}
\nocite{Braess1997}
\nocite{Beckie1993}
\nocite{Brenner1992}
\nocite{Wagner95}

\bibliographystyle{unsrt}
\bibliography{FEMmultigrid}
\end{document}
