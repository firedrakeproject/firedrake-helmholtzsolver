\documentclass[12pt]{article}
\usepackage{amsmath,amssymb}
\usepackage[cm]{fullpage}
\renewcommand{\vec}[1]{\boldsymbol{#1}}
\newcommand{\grad}{\operatorname{grad}}
\newcommand{\divergence}{\operatorname{div}}
\newcommand{\intOmega}[1]{\langle#1\rangle_{\Omega}}
\newcommand{\Vpressure}{\mathbb{V}_2}
\newcommand{\Vvelocity}{\mathbb{V}_1}
\newcommand{\VvelocityTr}{\mathbb{V}^*_1}
\newcommand{\indexSet}{\mathcal{I}}
\newcommand{\grid}{\mathcal{T}}
\newcommand{\Id}{\operatorname{Id}}
\usepackage{algorithm,algorithmic}
\title{One dimensional mixed system}
\author{Eike M\"{u}ller, University of Bath}
\begin{document}
\maketitle
%%%%%%%%%%%%%%%%%%%%%%%%%%%%%%%%%%%%%%%%%%
\section{Basis functions}
%%%%%%%%%%%%%%%%%%%%%%%%%%%%%%%%%%%%%%%%%%
Consider the mixed system arising from
\begin{equation}
  \begin{aligned}
    \phi + \omega \operatorname{div}(u) &= r_\phi \\
    u + \omega \operatorname{grad}\phi &= r_u
  \end{aligned}
\end{equation}
in 1d, discretised with $DG(n)$ for pressure and $P2$  for velocity\footnote{$P2$ are the 2nd order polynomials which are continuous across cells}.
Consider the following basis functions on the reference element $[0,1]$:
\begin{itemize}
  \item Pressure:\begin{equation}\phi_j(x) = \begin{cases} 1 & \text{for $j=0$}\\p_j(x) & \text{for $j=1,\dots,n$}\end{cases}\end{equation} where $p_j$ is a polynomial such that $\int_{0}^1p_j(x)\;dx = 0$
  \item Velocity: $u_+(x) = x$, $u_-(x) = 1-x$, $\tilde{u} = 4x(1-x)$
\end{itemize}
Denote the dof- vectors of functions spanned by the basis functions by $\vec{\Phi}_0$, $\tilde{\vec{\Phi}}$, $\vec{U}_0$ and $\tilde{\vec{U}}$ respectively.
By ordering the dofs such that the cell-index runs fastest, the mass matrices look like this:
\begin{xalignat}{2}
  M_\phi &= \Id
\begin{pmatrix}
  h^2\Id_{N\times N} & 0 \\
  0 & \tilde{M}_{\phi}
\end{pmatrix}
&
  M_u &= 
\begin{pmatrix}
  M_u^{(0)} & M_u' \\
  M_u' & \tilde{M}_u
\end{pmatrix}
\end{xalignat}
Here $N$ is the number of cells, $h$ the cell size, and I assume periodic BCs. Then $\tilde{M_\phi}$ is $(nN)\times (nN)$ and $M_u^{\pm}$, $M_u'$, $M_u'$ and $\tilde{M}_u$ are all $N\times N$.
It can also be shown very easily that the divergence matrix $\int_E \operatorname{div}(u)\phi\;dx$ only couples $\vec{\Phi}_0 \leftrightarrow \vec{U}_0$ and $\tilde{\vec{\Phi}} \leftrightarrow \tilde{\vec{U}}$.
%%%%%%%%%%%%%%%%%%%%%%%%%%%%%%%%%%%%%%%%%%
\section{Mixed system}
%%%%%%%%%%%%%%%%%%%%%%%%%%%%%%%%%%%%%%%%%%
In terms of the dofs-vectors, the mixed system looks like this (note the ordering of the full dof-vector):
\begin{equation}
\begin{pmatrix}
  h^2\Id_{N\times N} & B_0 & 0 & 0 \\
  B_0^T & M_u^{(0)} & 0 & M_u' \\
   0 & 0 & \tilde{M}_\phi & \tilde{B} \\
   0 & M_u' & \tilde{B}^T & \tilde{M}_u
\end{pmatrix}
\begin{pmatrix}
  \vec{\Phi}_0 \\
  \vec{U}_0 \\
  \tilde{\vec{\Phi}} \\
  \tilde{\vec{U}}
\end{pmatrix}
\label{eqn:System}
\end{equation}
The $(\vec{\Phi}_0,\vec{U}_0)\equiv \vec{X}_0$ dofs only couple to the $(\tilde{\vec{\Phi}},\tilde{\vec{U}})\equiv \tilde{\vec{X}}$ dofs through the matrix $M_u'$. Also, the lower right of the full system matrix in Eqn. (\ref{eqn:System}), which describes the self-coupling of the $(\tilde{\vec{\Phi}},\tilde{\vec{U}})$ dofs, can be brought into block-diagonal form since the $\tilde{u}$ basis function is not shared between cells. So, after a suitable redordering of the $\tilde{\vec{X}}$ dofs, the system becomes
\begin{equation}
\begin{pmatrix}
A & C \\
C^T & D  
\end{pmatrix}
\begin{pmatrix}
\vec{X}_0 \\
\tilde{\vec{X}}
\end{pmatrix}
=
\begin{pmatrix}
\vec{R}_0 \\
\tilde{\vec{R}}
\end{pmatrix}
\label{eqn:finalSystem}
\end{equation}
where $D$ is block-diagonal, so can be inverted very easily cell-by cell. The matrix $A$ describes the lowest order $DG(0)$ - $L1$ mixed system which we would solve with the help of the $DG(0)$ multigrid preconditioner (possible with velocity mass-lumping in the velocity, but in principle it can be solved with the full velocity mass matrix). So in summary, I assume that we can solve the systems $A\vec{X}_0 = \vec{B}_0$ and $D\tilde{\vec{X}} = \tilde{\vec{B}}$ for $\vec{X}_0$ and $\tilde{\vec{X}}$.
%%%%%%%%%%%%%%%%%%%%%%%%%%%%%%%%%%%%%%%%%%
\section{Schur complement}
%%%%%%%%%%%%%%%%%%%%%%%%%%%%%%%%%%%%%%%%%%
With the help of the Schur complement $A-CD^{-1}C^T$ the inverse of the matrix in Eqn. (\ref{eqn:finalSystem}) can be written explicitly as
\begin{equation}
  \begin{pmatrix}
A & C \\
C^T & D  
\end{pmatrix}^{-1}
=
\begin{pmatrix}
  \Id & 0 \\ -D^{-1}C^T \Id 
\end{pmatrix}
\begin{pmatrix}
\left(A-CD^{-1}C^T\right)^{-1}  & 0 \\
0 & D^{-1}  
\end{pmatrix}
\begin{pmatrix}
\Id & -CD^{-1}\\ 0 & \Id 
\end{pmatrix}
\end{equation}
Also note that due to the particular structure of $C$ and because $D$ is block-diagonal, the matrix $CD^{-1}C^T$ looks like this:
\begin{equation}
CD^{-1}C^T =
\begin{pmatrix}
  0 & 0 \\
  0 & M_u' D_u M_u'
\end{pmatrix},
\end{equation}
where $D_u$ is a diagonal matrix. It follows that the matrix in the lower-order space is
\begin{equation}
A-CD^{-1}C^T
=
\begin{pmatrix}
h^2\Id_{N\times N} & B_0  \\
  B_0^T & M_u^{(0)} - M_u' D_u M_u'
\end{pmatrix}
\end{equation}
i.e. all we have to do is add a correction to the velocity mass matrix.
%%%%%%%%%%%%%%%%%%%%%%%%%%%%%%%%%%%%%%%%%%
\section{Iteration}
%%%%%%%%%%%%%%%%%%%%%%%%%%%%%%%%%%%%%%%%%%
This suggests that a possible preconditioner is
\begin{equation}
P=
\begin{pmatrix}
\left(A-CD^{-1}C^T\right)^{-1}  & 0 \\
0 & D^{-1}  
\end{pmatrix}
\end{equation}
which could be used for a Richardson iteration or in a Krylov-subspace method.

For example, it leads to the following preconditioned Richardson iteration:
\begin{equation}
\begin{pmatrix}
  \vec{X}^{(k+1)}_0\\ \tilde{\vec{X}}^{(k+1)}
\end{pmatrix}
= 
\begin{pmatrix}
\vec{X}^{(k)}_0\\ \tilde{\vec{X}}^{(k)}
\end{pmatrix}
+
\mu
\begin{pmatrix}
\left(A-CD^{-1}C^T\right)^{-1} & 0 \\ 0 & D^{-1}
\end{pmatrix}
\left[
\begin{pmatrix}
  \vec{R}_0 \\ \tilde{\vec{R}}
\end{pmatrix}
-
\begin{pmatrix}
A & C \\ C^T & D
\end{pmatrix}
\begin{pmatrix}
\vec{X}^{(k)}_0 \\ \tilde{\vec{X}}^{(k)}
\end{pmatrix}
\right]
\end{equation}
Maybe it is also sufficient to use
\begin{equation}
P=
\begin{pmatrix}
A^{-1}  & 0 \\
0 & D^{-1}  
\end{pmatrix}
\end{equation}

\end{document}