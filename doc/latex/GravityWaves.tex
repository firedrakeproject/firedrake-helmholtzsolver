\documentclass[10pt]{article}
\usepackage[cm]{fullpage}
\usepackage{amsmath,amssymb}
\usepackage{hyperref}
\newcommand{\Uspace}{\mathbb{U}}
\newcommand{\Vspace}{\mathbb{V}}
\newcommand{\Wspace}{\mathbb{W}}
\newcommand{\Hdiv}{\texttt{HDiv}}
\newcommand{\Hcurl}{\texttt{HCurl}}
\renewcommand{\vec}[1]{\boldsymbol{#1}}
\newcommand{\zhat}{\hat{\vec{z}}}
\newcommand{\ddt}[1]{\frac{\partial #1}{\partial t}}
\newcommand{\ddtdisc}[1]{\frac{{#1}^{(t+\Delta t)}-{#1}^{(t)}}{\Delta t}}
\newcommand{\tavg}[1]{\frac{{#1}^{(t+\Delta t)}+{#1}^{(t)}}{2}}
\title{Matrix-free 3d solver for linear gravity wave test case}
\date{\today}
\author{Eike Hermann M\"{u}ller, Department of Mathematical Sciences, University of Bath}
\begin{document}
\maketitle
%%%%%%%%%%%%%%%%%%%%%%%%%%%%%%%%%%%%%%%%%%%%%%%
\section{Problem formulation}
%%%%%%%%%%%%%%%%%%%%%%%%%%%%%%%%%%%%%%%%%%%%%%%
%%%%%%%%%%%%%%%%%%%%%%%%%%%%%%%%%%%%%%%%%%%%%%%
\subsection{Continuous equations}
%%%%%%%%%%%%%%%%%%%%%%%%%%%%%%%%%%%%%%%%%%%%%%%
Consider the following linear gravity wave problem \cite{McRae2014}
\begin{xalignat}{3}
   \ddt{\vec{u}} &= \nabla p + b \zhat, &
   \ddt{p} &= -c^2 \nabla\cdot \vec{u}, &
   \ddt{b} &= -N^2\vec{u}\cdot\zhat.
\label{eqn:ContinuousEquations}
\end{xalignat}
We assume that both the speed of sound $c$ and the buoyancy frequency $N$ are constant and enforce the condition $\vec{u}\cdot\vec{n}=0$.

%%%%%%%%%%%%%%%%%%%%%%%%%%%%%%%%%%%%%%%%%%%%%%%
\subsection{Finite element discretisation}
%%%%%%%%%%%%%%%%%%%%%%%%%%%%%%%%%%%%%%%%%%%%%%%
%%%%%%%%%%%%%%%%%%%%%%%%%%%%%%%%%%%%%%%%%%%%%%%
\paragraph{Function spaces.}
%%%%%%%%%%%%%%%%%%%%%%%%%%%%%%%%%%%%%%%%%%%%%%%
We have the following de Rham complexes in one, two and three dimensions:
\begin{xalignat}{3}
  \Vspace_0 \overset{\partial_z}{\rightarrow}\Vspace_1,&
  &\Uspace_0 \overset{\nabla^{\perp}}{\rightarrow}
\Uspace_1 \overset{\nabla\cdot}{\rightarrow}\Uspace_2, &
  &\Wspace_0 \overset{\nabla}{\rightarrow} \Wspace_1 \overset{\nabla\times}{\rightarrow} \Wspace_2\overset{\nabla\cdot}{\rightarrow} \Wspace_3
\end{xalignat}
with
\begin{equation}
 \begin{aligned}
  \Wspace_0 &= \Uspace_0\otimes\Vspace_0,\\
  \Wspace_1 &= \Hcurl(\Uspace_1\otimes\Vspace_0)\oplus
\Hcurl(\Uspace_0\otimes\Vspace_1),\\
  \Wspace_2 &= \Hdiv(\Uspace_2\otimes\Vspace_0)\oplus
\Hdiv(\Uspace_1\otimes\Vspace_1),\\
  \Wspace_3 &= \Uspace_2\otimes\Uspace_1,
 \end{aligned}
\end{equation}
Also denote $\Wspace_2^0$ denoting the space of all functions in $\Wspace_2$ which satisfy the boundary condition $\vec{u}\cdot\vec{n}=0$. We write $\Wspace_2^v \equiv \Uspace_2\otimes\Vspace_0$ for the vertical component of the velocity space. Then the fields in (\ref{eqn:ContinuousEquations}) live in the following spaces:
\begin{xalignat}{3}
  \vec{u} &\in \Wspace_2^0,&
  b &\in \Wspace_2^v, &
  p &\in \Wspace_3.
\end{xalignat}
%%%%%%%%%%%%%%%%%%%%%%%%%%%%%%%%%%%%%%%%%%%%%%%
\paragraph{Discrete equations.}
%%%%%%%%%%%%%%%%%%%%%%%%%%%%%%%%%%%%%%%%%%%%%%%
Multiply with basis functions and use the implicit midpoint rule:
\begin{equation}
 \begin{aligned}
  &\langle\vec{w},\ddtdisc{\vec{u}}\rangle
  - \langle\nabla\cdot\vec{w},\tavg{p}\rangle
  - \langle\vec{w},\tavg{b}\zhat\rangle = 0\\[1ex]
  &\langle\phi,\ddtdisc{p}\rangle
  +c^2 \langle\phi,\nabla\cdot\tavg{\vec{u}}\rangle
  = 0\\[1ex]
  &\langle\gamma,\ddtdisc{b}\rangle
  + N^2\langle\gamma,\tavg{\vec{u}}\cdot\zhat\rangle
  = 0
 \end{aligned}
\end{equation}
Define $\delta p\equiv p^{(t+\Delta t)}-p^{(t)}$, $p_0\equiv p^{(t)}$ etc. and note that $\tavg{p}=\frac{\delta p}{2}+p_0$ to obtain an equation for the increments
\begin{equation}
 \begin{aligned}
  &\langle\vec{w},\delta\vec{u}\rangle
  - \langle\nabla\cdot\vec{w},\delta p\rangle
  - \langle\vec{w},\delta b\zhat\rangle
  = \Delta t \langle\nabla\cdot\vec{w},p_0\rangle
  + \Delta t \langle\vec{w},b_0\zhat\rangle
  \equiv r_u
 \\[1ex]
  &\langle\phi,\delta p\rangle
  +c^2 \langle\phi,\nabla\cdot\delta\vec{u}\rangle
  = -\Delta tc^2\langle\phi,\nabla\cdot\vec{u}_0\rangle
  \equiv r_\phi
  \\[1ex]
  &\langle\gamma,\delta b\rangle
  + N^2\langle\gamma,\delta\vec{u}\cdot\zhat\rangle
  = -\Delta t N^2\langle\gamma,\vec{u}_0\cdot\zhat\rangle
  \equiv r_b\label{eqn:Increments}
 \end{aligned}
\end{equation}
%%%%%%%%%%%%%%%%%%%%%%%%%%%%%%%%%%%%%%%%%%%%%%%
\paragraph{Matrix formulation.}
%%%%%%%%%%%%%%%%%%%%%%%%%%%%%%%%%%%%%%%%%%%%%%%
By introducing the mass matrices
\begin{xalignat}{3}
  \left(M_u\right)_{ij} &\equiv \langle \vec{w}_i\cdot\vec{w}_j\rangle,&
  \left(M_p\right)_{ij} &\equiv \langle \phi_i,\phi_j\rangle, &
  \left(M_b\right)_{ij} &\equiv \langle \gamma_i,\gamma_j \rangle
\end{xalignat}
and the derivative- and projection operators
\begin{xalignat}{2}
  \left(D^T\right)_{ij} &\equiv \langle\nabla\cdot\vec{w}_i,\phi_j\rangle, &
  \left(Q\right)_{ij} &\equiv \langle\vec{w}_i,\gamma_j\zhat\rangle,
\end{xalignat}
equation (\ref{eqn:Increments}) can be written as a matrix equation for the dof-vectors $\vec{U}$ (velocity), $\vec{P}$ (pressure) and $\vec{B}$ (buoyancy):
\begin{equation}
\begin{pmatrix}
  M_u & 
    -\frac{\Delta t}{2}D^T & 
    -\frac{\Delta t}{2}Q\\[1ex]
  \frac{\Delta t}{2}c^2D & M_p & 0\\[1ex]
  \frac{\Delta t}{2}N^2Q^T & 0 & M_b
\end{pmatrix}
\begin{pmatrix}
  \vec{U}\\[1ex]\vec{P}\\[1ex]\vec{B}
\end{pmatrix}
=
\begin{pmatrix}
  \vec{R}_u\\[1ex]\vec{R}_p\\[1ex]\vec{R}_b
\end{pmatrix}.
\end{equation}
Use the third equation to eliminate the buoyancy  $\vec{B}=M_b^{-1}\vec{R}_b-\frac{\Delta t}{2}N^2M_b^{-1} Q^T\vec{U}$ and insert it into the velocity equation to obtain a mixed system for pressure and velocity:
\begin{equation}
  A\begin{pmatrix}\vec{P}\\[1ex]\vec{U}\end{pmatrix}
  \equiv
  \begin{pmatrix}
   M_p & \frac{\Delta t}{2}c^2 D \\[1ex]
   -\frac{\Delta t}{2}D^T & \tilde{M}_u
  \end{pmatrix}
  \begin{pmatrix}\vec{P}\\[1ex]\vec{U}\end{pmatrix}
 =
\begin{pmatrix}\vec{R}_p\\[1ex]\tilde{\vec{R}}_u\end{pmatrix}
\qquad\text{with}\quad 
\tilde{M}_u \equiv M_u + \omega^2_N QM_b^{-1}Q^T
\quad\text{and}\quad \omega_N \equiv \frac{\Delta t}{2}N\label{eqn:PressureVelocitySystem}
\end{equation}
In the absence of orography, $Q=Q^T=M_b$. In any case, both $M_b$ and $\tilde{M}_u$ are well conditioned, so can be inverted with a small number of Richardson- or CG- iterations.
%%%%%%%%%%%%%%%%%%%%%%%%%%%%%%%%%%%%%%%%%%%%%%%
\section{Schur complement Preconditioner}
%%%%%%%%%%%%%%%%%%%%%%%%%%%%%%%%%%%%%%%%%%%%%%%
The inverse of the operator $A$ in (\ref{eqn:PressureVelocitySystem}) is given by
\begin{equation}
  A^{-1} = 
\begin{pmatrix}
  1 & 0 \\[1ex]
  \frac{\Delta t}{2}\tilde{M}_u^{-1} D^T & 1
\end{pmatrix}
\begin{pmatrix}
  H^{-1} & 0 \\[1ex] 0 & \tilde{M}_u^{-1}
\end{pmatrix}
\begin{pmatrix}
  1 & -\frac{\Delta t}{2}D\tilde{M}_u^{-1} \\[1ex]
  0 & 1
\end{pmatrix}
\end{equation}
with the (positive-definite) ``Helmholtz'' operator
\begin{equation}
  H \equiv M_p + \omega_c^2 D\tilde{M}_u^{-1} D^T
\qquad\text{where}\quad \omega_c \equiv \frac{\Delta t}{2}c.
\end{equation}
Precondition this operator with a similar operator $\hat{H}$ which only contains the vertical couplings, i.e. assume that $p\in \Uspace_2\otimes \Vspace_1$, $\vec{u}\in \Hdiv(\Uspace_2\otimes\Vspace_0)$. Denote the basis functions of $\Uspace_2$ as $\sigma_i(\vec{x})$, the basis functions of $\Vspace_0$ as $\alpha_k(\vec{x})$ and the basis functions of $\Vspace_1$ as $\beta_k(\vec{x})$. Then we have
\begin{equation}
  \hat{H} = M_{\phi} + \frac{\omega_c^2}{1+\omega_N^2}D_z\left(M^*_{u}\right)^{-1} D_z^T
\label{eqn:Preconditioner}
\end{equation}
where we defined\footnote{the notation is a bit sloppy, the velocity space os vector-valued, so the correct contravariant Piola transfroms have to be included}
\begin{equation}
 \begin{aligned}
   \left(M_\phi\right)_{ij,k\ell} &\equiv 
   \int \sigma_i(\vec{x})\sigma_j(\vec{x})
   \beta_k(\vec{x})\beta_\ell(\vec{x})
   \;d\vec{x}\\[1ex]
   \left(M^*_{u}\right)_{ij,k\ell} &\equiv 
   \int \sigma_i(\vec{x})\sigma_j(\vec{x})
   \alpha_k(\vec{x})\alpha_\ell(\vec{x})
   \;d\vec{x}\\[1ex]
   \left(D_z^T\right)_{ij,k\ell} &\equiv 
   \int \sigma_i(\vec{x})\sigma_j(\vec{x})
   (\partial_z\alpha_k(\vec{x}))\beta_\ell(\vec{x})
   \;d\vec{x}.
 \end{aligned}
\end{equation}
Here $i,j$ are the indices of the horizontal dofs in one horizontal cell (since $\Uspace_2$ is a DG space, there will be no couplings between different columns) and $k,\ell$ the indices of \textit{all} $\Vspace_0$ or $\Vspace_1$ dofs in one column.   Denoting the dof-vector in a column as $\vec{X}$, we have, for example
\begin{equation}
  \left(A\vec{X}\right)_{ik} \equiv \sum_{j\ell} A_{ij,k\ell} X_{j\ell}.
\end{equation}
The products of two matrices in (\ref{eqn:Preconditioner}) are defined as
\begin{equation}
  \left(AB\right)_{ij,k\ell} \equiv \sum_{r,s} A_{ir,ks}B_{rj,s\ell}
\end{equation}
While $M_u^*$ is a sparse, block-banded matrix, the inverse $M_u^*$ is dense and has the be approximated somehow (maybe using a SPAI?).
\bibliographystyle{plain}
\bibliography{formulation}
\end{document}